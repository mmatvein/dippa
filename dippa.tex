% Lines starting with a percent sign (%) are comments. LaTeX will
% not process those lines. Similarly, everything after a percent
% sign in a line is considered a comment. To produce a percent sign
% in the output, write \% (backslash followed by the percent sign).
% ==================================================================
% Usage instructions:
% ------------------------------------------------------------------
% The file is heavily commented so that you know what the various
% commands do. Feel free to remove any comments you don't need from
% your own copy. When redistributing the example thesis file, please
% retain all the comments for the benefit of other thesis writers!
% ==================================================================
% Compilation instructions:
% ------------------------------------------------------------------
% Use pdflatex to compile! Input images are expected as PDF files.
% Example compilation:
% ------------------------------------------------------------------
% > pdflatex thesis-example.tex
% > bibtex thesis-example
% > pdflatex thesis-example.tex
% > pdflatex thesis-example.tex
% ------------------------------------------------------------------
% You need to run pdflatex multiple times so that all the cross-references
% are fixed. pdflatex will tell you if you need to re-run it (a warning
% will be issued)
% ------------------------------------------------------------------
% Compilation has been tested to work in ukk.cs.hut.fi and kosh.hut.fi
% - if you have problems of missing .sty -files, then the local LaTeX
% environment does not have all the required packages installed.
% For example, when compiling in vipunen.hut.fi, you get an error that
% tikz.sty is missing - in this case you must either compile somewhere
% else, or you cannot use TikZ graphics in your thesis and must therefore
% remove or comment out the tikz package and all the tikz definitions.
% ------------------------------------------------------------------

% General information
% ==================================================================
% Package documentation:
%
% The comments often refer to package documentation. (Almost) all LaTeX
% packages have documentation accompanying them, so you can read the
% package documentation for further information. When a package 'xxx' is
% installed to your local LaTeX environment (the document compiles
% when you have \usepackage{xxx} and LaTeX does not complain), you can
% find the documentation somewhere in the local LaTeX texmf directory
% hierarchy. In ukk.cs.hut.fi, this is /usr/texlive/2008/texmf-dist,
% and the documentation for the titlesec package (for example) can be
% found at /usr/texlive/2008/texmf-dist/doc/latex/titlesec/titlesec.pdf.
% Most often the documentation is located as a PDF file in
% /usr/texlive/2008/texmf-dist/doc/latex/xxx, where xxx is the package name;
% however, documentation for TikZ is in
% /usr/texlive/2008/texmf-dist/doc/latex/generic/pgf/pgfmanual.pdf
% (this is because TikZ is a front-end for PGF, which is meant to be a
% generic portable graphics format for LaTeX).
% You can try to look for the package manual using the ``find'' shell
% command in Linux machines; the find databases are up-to-date at least
% in ukk.cs.hut.fi. Just type ``find xxx'', where xxx is the package
% name, and you should find a documentation file.
% Note that in some packages, the documentation is in the DVI file
% format. In this case, you can copy the DVI file to your home directory,
% and convert it to PDF with the dvipdfm command (or you can read the
% DVI file directly with a DVI viewer).
%
% If you can't find the documentation for a package, just try Googling
% for ``latex packagename''; most often you can get a direct link to the
% package manual in PDF format.
% ------------------------------------------------------------------


% Document class for the thesis is report
% ------------------------------------------------------------------
% You can change this but do so at your own risk - it may break other things.
% Note that the option pdftext is used for pdflatex; there is no
% pdflatex option.
% ------------------------------------------------------------------
\documentclass[12pt,a4paper,oneside,pdftex]{report}

% The input files (tex files) are encoded with the latin-1 encoding
% (ISO-8859-1 works). Change the latin1-option if you use UTF8
% (at some point LaTeX did not work with UTF8, but I'm not sure
% what the current situation is)
\usepackage[utf8]{inputenc}
% OT1 font encoding seems to work better than T1. Check the rendered
% PDF file to see if the fonts are encoded properly as vectors (instead
% of rendered bitmaps). You can do this by zooming very close to any letter
% - if the letter is shown pixelated, you should change this setting
% (try commenting out the entire line, for example).
\usepackage[OT1]{fontenc}
% The babel package provides hyphenating instructions for LaTeX. Give
% the languages you wish to use in your thesis as options to the babel
% package (as shown below). You can remove any language you are not
% going to use.
% Examples of valid language codes: english (or USenglish), british,
% finnish, swedish; and so on.
\usepackage[finnish,english]{babel}


% Font selection
% ------------------------------------------------------------------
% The default LaTeX font is a very good font for rendering your
% thesis. It is a very professional font, which will always be
% accepted.
% If you, however, wish to spicen up your thesis, you can try out
% these font variants by uncommenting one of the following lines
% (or by finding another font package). The fonts shown here are
% all fonts that you could use in your thesis (not too silly).
% Changing the font causes the layouts to shift a bit; you many
% need to manually adjust some layouts. Check the warning messages
% LaTeX gives you.
% ------------------------------------------------------------------
% To find another font, check out the font catalogue from
% http://www.tug.dk/FontCatalogue/mathfonts.html
% This link points to the list of fonts that support maths, but
% that's a fairly important point for master's theses.
% ------------------------------------------------------------------
% <rant>
% Remember, there is no excuse to use Comic Sans, ever, in any
% situation! (Well, maybe in speech bubbles in comics, but there
% are better options for those too)
% </rant>

% \usepackage{palatino}
% \usepackage{tgpagella}



% Optional packages
% ------------------------------------------------------------------
% Select those packages that you need for your thesis. You may delete
% or comment the rest.

% Natbib allows you to select the format of the bibliography references.
% The first example uses numbered citations:
\usepackage[square,sort&compress,numbers]{natbib}
% The second example uses author-year citations.
% If you use author-year citations, change the bibliography style (below);
% acm style does not work with author-year citations.
% Also, you should use \citet (cite in text) when you wish to refer
% to the author directly (\citet{blaablaa} said blaa blaa), and
% \citep when you wish to refer similarly than with numbered citations
% (It has been said that blaa blaa~\citep{blaablaa}).
% \usepackage[square]{natbib}

% The alltt package provides an all-teletype environment that acts
% like verbatim but you can use LaTeX commands in it. Uncomment if
% you want to use this environment.
% \usepackage{alltt}

% The eurosym package provides a euro symbol. Use with \euro{}
\usepackage{eurosym}

% HILIGHTS
\usepackage{soul}

% Verbatim provides a standard teletype environment that renderes
% the text exactly as written in the tex file. Useful for code
% snippets (although you can also use the listings package to get
% automatic code formatting).
\usepackage{verbatim}

% The listing package provides automatic code formatting utilities
% so that you can copy-paste code examples and have them rendered
% nicely. See the package documentation for details.
% \usepackage{listings}

% The fancuvrb package provides fancier verbatim environments
% (you can, for example, put borders around the verbatim text area
% and so on). See package for details.
% \usepackage{fancyvrb}

% Supertabular provides a tabular environment that can span multiple
% pages.
%\usepackage{supertabular}
% Longtable provides a tabular environment that can span multiple
% pages. This is used in the example acronyms file.
\usepackage{longtable}

% The fancyhdr package allows you to set your the page headers
% manually, and allows you to add separator lines and so on.
% Check the package documentation.
% \usepackage{fancyhdr}

% Subfigure package allows you to use subfigures (i.e. many subfigures
% within one figure environment). These can have different labels and
% they are numbered automatically. Check the package documentation.
\usepackage{subfigure}

% The titlesec package can be used to alter the look of the titles
% of sections, chapters, and so on. This example uses the ``medium''
% package option which sets the titles to a medium size, making them
% a bit smaller than what is the default. You can fine-tune the
% title fonts and sizes by using the package options. See the package
% documentation.
\usepackage[medium]{titlesec}

% The TikZ package allows you to create professional technical figures.
% The learning curve is quite steep, but it is definitely worth it if
% you wish to have really good-looking technical figures.
\usepackage{tikz}
% You also need to specify which TikZ libraries you use
\usetikzlibrary{positioning}
\usetikzlibrary{calc}
\usetikzlibrary{arrows}
\usetikzlibrary{decorations.pathmorphing,decorations.markings}
\usetikzlibrary{shapes}
\usetikzlibrary{patterns}


% The aalto-thesis package provides typesetting instructions for the
% standard master's thesis parts (abstracts, front page, and so on)
% Load this package second-to-last, just before the hyperref package.
% Options that you can use:
%   mydraft - renders the thesis in draft mode.
%             Do not use for the final version.
%   doublenumbering - [optional] number the first pages of the thesis
%                     with roman numerals (i, ii, iii, ...); and start
%                     arabic numbering (1, 2, 3, ...) only on the
%                     first page of the first chapter
%   twoinstructors  - changes the title of instructors to plural form
%   twosupervisors  - changes the title of supervisors to plural form
\usepackage[mydraft]{aalto-thesis}
%\usepackage[mydraft,doublenumbering]{aalto-thesis}
%\usepackage{aalto-thesis}


% Hyperref
% ------------------------------------------------------------------
% Hyperref creates links from URLs, for references, and creates a
% TOC in the PDF file.
% This package must be the last one you include, because it has
% compatibility issues with many other packages and it fixes
% those issues when it is loaded.
\RequirePackage[pdftex]{hyperref}
% Setup hyperref so that links are clickable but do not look
% different
\hypersetup{colorlinks=false,raiselinks=false,breaklinks=true}
\hypersetup{pdfborder={0 0 0}}
\hypersetup{bookmarksnumbered=true}
% The following line suggests the PDF reader that it should show the
% first level of bookmarks opened in the hierarchical bookmark view.
\hypersetup{bookmarksopen=true,bookmarksopenlevel=1}
% Hyperref can also set up the PDF metadata fields. These are
% set a bit later on, after the thesis setup.


% Thesis setup
% ==================================================================
% Change these to fit your own thesis.
% \COMMAND always refers to the English version;
% \FCOMMAND refers to the Finnish version; and
% \SCOMMAND refers to the Swedish version.
% You may comment/remove those language variants that you do not use
% (but then you must not include the abstracts for that language)
% ------------------------------------------------------------------
% If you do not find the command for a text that is shown in the cover page or
% in the abstract texts, check the aalto-thesis.sty file and locate the text
% from there.
% All the texts are configured in language-specific blocks (lots of commands
% that look like this: \renewcommand{\ATCITY}{Espoo}.
% You can just fix the texts there. Just remember to check all the language
% variants you use (they are all there in the same place).
% ------------------------------------------------------------------
\newcommand{\TITLE}{The Design and Implementation of a Virtual Reality Toolkit}
\newcommand{\FTITLE}{}
\newcommand{\STITLE}{}
\newcommand{\SUBTITLE}{}
\newcommand{\FSUBTITLE}{}
\newcommand{\SSUBTITLE}{}
\newcommand{\DATE}{}
\newcommand{\FDATE}{}
\newcommand{\SDATE}{}

% Supervisors and instructors
% ------------------------------------------------------------------
% If you have two supervisors, write both names here, separate them with a
% double-backslash (see below for an example)
% Also remember to add the package option ``twosupervisors'' or
% ``twoinstructors'' to the aalto-thesis package so that the titles are in
% plural.
% Example of one supervisor:
%\newcommand{\SUPERVISOR}{Professor Antti Ylä-Jääski}
%\newcommand{\FSUPERVISOR}{Professori Antti Ylä-Jääski}
%\newcommand{\SSUPERVISOR}{Professor Antti Ylä-Jääski}
% Example of twosupervisors:
\newcommand{\SUPERVISOR}{Professor Perttu Hämäläinen}
\newcommand{\FSUPERVISOR}{Professori Perttu Hämäläinen}
\newcommand{\SSUPERVISOR}{Professor Perttu Hämäläinen}

% If you have only one instructor, just write one name here
\newcommand{\INSTRUCTOR}{Tuukka Takala M.Sc. (Tech.)}
\newcommand{\FINSTRUCTOR}{Diplomi-insinööri Tuukka Takala}
%\newcommand{\SINSTRUCTOR}{Diplomingenjör Olli Ohjaaja}
% If you have two instructors, separate them with \\ to create linefeeds
% \newcommand{\INSTRUCTOR}{Olli Ohjaaja M.Sc. (Tech.)\\
%  Elli Opas M.Sc. (Tech)}
%\newcommand{\FINSTRUCTOR}{Diplomi-insinööri Olli Ohjaaja\\
%  Diplomi-insinööri Elli Opas}
%\newcommand{\SINSTRUCTOR}{Diplomingenjör Olli Ohjaaja\\
%  Diplomingenjör Elli Opas}

% If you have two supervisors, it is common to write the schools
% of the supervisors in the cover page. If the following command is defined,
% then the supervisor names shown here are printed in the cover page. Otherwise,
% the supervisor names defined above are used.
%\newcommand{\COVERSUPERVISOR}{}

% The same option is for the instructors, if you have multiple instructors.
% \newcommand{\COVERINSTRUCTOR}{Olli Ohjaaja M.Sc. (Tech.), Aalto University\\
%  Elli Opas M.Sc. (Tech), Aalto SCI}


% Other stuff
% ------------------------------------------------------------------
\newcommand{\PROFESSORSHIP}{}
\newcommand{\FPROFESSORSHIP}{}
\newcommand{\SPROFESSORSHIP}{}
% Professorship code is the same in all languages
\newcommand{\PROFCODE}{}
\newcommand{\KEYWORDS}{}
\newcommand{\FKEYWORDS}{}
\newcommand{\LANGUAGE}{English}
\newcommand{\FLANGUAGE}{Englanti}
\newcommand{\SLANGUAGE}{Engelska}

% Author is the same for all languages
\newcommand{\AUTHOR}{Mikael Matveinen}


% Currently the English versions are used for the PDF file metadata
% Set the PDF title
\hypersetup{pdftitle={\TITLE\ \SUBTITLE}}
% Set the PDF author
\hypersetup{pdfauthor={\AUTHOR}}
% Set the PDF keywords
\hypersetup{pdfkeywords={\KEYWORDS}}
% Set the PDF subject
\hypersetup{pdfsubject={Master's Thesis}}


% Layout settings
% ------------------------------------------------------------------

% When you write in English, you should use the standard LaTeX
% paragraph formatting: paragraphs are indented, and there is no
% space between paragraphs.
% When writing in Finnish, we often use no indentation in the
% beginning of the paragraph, and there is some space between the
% paragraphs.

% If you write your thesis Finnish, uncomment these lines; if
% you write in English, leave these lines commented!
% \setlength{\parindent}{0pt}
% \setlength{\parskip}{1ex}

% Use this to control how much space there is between each line of text.
% 1 is normal (no extra space), 1.3 is about one-half more space, and
% 1.6 is about double line spacing.
% \linespread{1} % This is the default
% \linespread{1.3}

% Bibliography style
% acm style gives you a basic reference style. It works only with numbered
% references.
\bibliographystyle{acm}
% Plainnat is a plain style that works with both numbered and name citations.
% \bibliographystyle{plainnat}


% Extra hyphenation settings
% ------------------------------------------------------------------
% You can list here all the files that are not hyphenated correctly.
% You can provide many \hyphenation commands and/or separate each word
% with a space inside a single command. Put hyphens in the places where
% a word can be hyphenated.
% Note that (by default) LaTeX will not hyphenate words that already
% have a hyphen in them (for example, if you write ``structure-modification
% operation'', the word structure-modification will never be hyphenated).
% You need a special package to hyphenate those words.
\hyphenation{di-gi-taa-li-sta yksi-suun-tai-sta}



% The preamble ends here, and the document begins.
% Place all formatting commands and such before this line.
% ------------------------------------------------------------------
\begin{document}
% This command adds a PDF bookmark to the cover page. You may leave
% it out if you don't like it...
\pdfbookmark[0]{Cover page}{bookmark.0.cover}
% This command is defined in aalto-thesis.sty. It controls the page
% numbering based on whether the doublenumbering option is specified
\startcoverpage

% Cover page
% ------------------------------------------------------------------
% Options: finnish, english, and swedish
% These control in which language the cover-page information is shown
\coverpage{english}


% Abstracts
% ------------------------------------------------------------------
% Include an abstract in the language that the thesis is written in,
% and if your native language is Finnish or Swedish, one in that language.

% Abstract in English
% ------------------------------------------------------------------
\thesisabstract{english}{
The introduction of game consoles with motion tracked controllers and depth sensing cameras has initiated a new wave of movement-based applications and games. These input devices paired with 3D display technologies that have recently become mainstream give hobbyists a chance of participating in the creation of virtual reality software at a much lower price point than was previously possible.

This thesis details the design and implementation of RUIS (Reality-based User Interface System), a virtual reality toolkit for Unity3D. The main goal of RUIS is to make the creation of VR applications as straightforward as possible. This is achieved by providing users with basic building blocks that solve many of the challenges encountered in the development of VR software. These building blocks provide a layer of abstraction which gives the developer the chance to focus on interaction methods and the application itself.

The two main tasks RUIS handles are input and display configuration management. The challenges involved in the management of input methods are for example the calibraton of coordinate systems, providing a clean interface for the usage of different types of devices in a standarized way and providing prebuilt modules that handle common VR tasks such as head tracking. The multitude of different virtual reality display configurations on the other hand require solutions for multi-display environments, head tracked views, projector keystoning correction and stereoscopic displays. RUIS implements these features and gives the developer a chance to create motion-based 3D applications easily inside the Unity Editor. 

Special attention during development was also given to the usability of RUIS - especially the idiomatic usage of Unity3D. This emphasis was put to the test on the Aalto University course T-111.5400 Virtual Reality, where the students used RUIS to create their applications. This thesis also discusses the results and lessons learned during the course. }


%A dissertation or thesis is a document submitted in support of candidature
%for a degree or professional qualification presenting the author's research and
%findings. In some countries/universities, the word thesis or a cognate is used
%as part of a bachelor's or master's course, while dissertation is normally
%applied to a doctorate, whilst, in others, the reverse is true.

%\fixme{Abstract text goes here (and this is an example how to use fixme).}
%Fixme is a command that helps you identify parts of your thesis that still
%require some work. When compiled in the custom \texttt{mydraft} mode, text
%parts tagged with fixmes are shown in bold and with fixme tags around them. When
%compiled in normal mode, the fixme-tagged text is shown normally (without
%special formatting). The draft mode also causes the ``Draft'' text to appear on
%the front page, alongside with the document compilation date. The custom
%\texttt{mydraft} mode is selected by the \texttt{mydraft} option given for the
%package \texttt{aalto-thesis}, near the top of the \texttt{thesis-example.tex}
%file.

%The thesis example file (\texttt{thesis-example.tex}), all the chapter content
%files (\texttt{1introduction.tex} and so on), and the Aalto style file
%(\texttt{aalto-thesis.sty}) are commented with explanations on how the Aalto
%thesis works. The files also contain some examples on how to customize various
%details of the thesis layout, and of course the example text works as an
%example in itself. Please read the comments and the example text; that should
%get you well on your way!

% Abstract in Finnish
% ------------------------------------------------------------------
\thesisabstract{finnish}{

}

% Abstract in Swedish
% ------------------------------------------------------------------
%\thesisabstract{swedish}{
%}


% Acknowledgements
% ------------------------------------------------------------------
% Select the language you use in your acknowledgements
\selectlanguage{english}

% Uncomment this line if you wish acknoledgements to appear in the
% table of contents
%\addcontentsline{toc}{chapter}{Acknowledgements}

% The star means that the chapter isn't numbered and does not
% show up in the TOC
\chapter*{Acknowledgements}

I wish to thank all students who use \LaTeX\ for formatting their theses,
because theses formatted with \LaTeX\ are just so nice.

Thank you, and keep up the good work!
\vskip 10mm

\noindent Espoo, \DATE
\vskip 5mm
\noindent\AUTHOR

% Acronyms
% ------------------------------------------------------------------
% Use \cleardoublepage so that IF two-sided printing is used
% (which is not often for masters theses), then the pages will still
% start correctly on the right-hand side.
\cleardoublepage
% Example acronyms are placed in a separate file, acronyms.tex
% \input{acronyms}

\addcontentsline{toc}{chapter}{Abbreviations and Acronyms}
\chapter*{Abbreviations and Acronyms}

% The longtable environment should break the table properly to multiple pages,
% if needed

\noindent
\begin{longtable}{@{}p{0.25\textwidth}p{0.7\textwidth}@{}}
DOF & Degrees of freedom \\
CAVE & Cave Automatic Virtual Environment
\end{longtable}


% Table of contents
% ------------------------------------------------------------------
\cleardoublepage
% This command adds a PDF bookmark that links to the contents.
% You can use \addcontentsline{} as well, but that also adds contents
% entry to the table of contents, which is kind of redundant.
% The text ``Contents'' is shown in the PDF bookmark.
\pdfbookmark[0]{Contents}{bookmark.0.contents}
\tableofcontents

% List of tables
% ------------------------------------------------------------------
% You only need a list of tables for your thesis if you have very
% many tables. If you do, uncomment the following two lines.
% \cleardoublepage
% \listoftables

% Table of figures
% ------------------------------------------------------------------
% You only need a list of figures for your thesis if you have very
% many figures. If you do, uncomment the following two lines.
% \cleardoublepage
% \listoffigures

% The following label is used for counting the prelude pages
\label{pages-prelude}
\cleardoublepage

%%%%%%%%%%%%%%%%% The main content starts here %%%%%%%%%%%%%%%%%%%%%
% ------------------------------------------------------------------
% This command is defined in aalto-thesis.sty. It controls the page
% numbering based on whether the doublenumbering option is specified
\startfirstchapter

% Add headings to pages (the chapter title is shown)
\pagestyle{headings}

% The contents of the thesis are separated to their own files.
% Edit the content in these files, rename them as necessary.
% ------------------------------------------------------------------

% \input{1introduction.tex}

\chapter{Introduction}
\label{chapter:intro}


% \input{2background.tex}


%%%%%%%%%%%%%%%%%%%%%%%%%%%%%%%%%%%%%%%%%%%%%%%%%%%%%%%%%%%%%%%%%%
%%%%%%%%%%%%%%%%%%%%%%%%%%%%%%%%%%%%%%%%%%%%%%%%%%%%%%%%%%%%%%%%%%


\chapter{Background}
\label{chapter:background}

This chapter briefly discusses the history of virtual reality and how it has come to the stage where a toolkit like RUIS makes it feasible for hobbyists to get involved.

\section{Brief History of Virtual Reality}
\label{section:historyofvr}

\section{Display environments}
\label{section:displayenvironments}

\subsection{Stereoscopic Displays and Projectors}
\label{subsection:displayenvironments:stereoscopic}


\subsection{Cave Automatic Virtual Environment (CAVE)}
\label{subsection:displayenvironments:cave}

\subsection{Head-mounted Displays}
\label{subsection:displayenvironments:hmd}

\section{Motion controllers}
\label{section:motioncontrollers}



\subsection{Depth-sensing Cameras}
\label{subsection:motion:kinect}

The Microsoft Kinect is a depth-sensing camera originally launched in November 2010. \hl{SOURCE}

\subsection{Wand Controllers}
\label{subsection:motion:move}

Wand controllers are input devices that are held in the hand like a wand. These devices can be used to point at virtual objects and manipulate them. Wand controllers usually include buttons at convenient locations to make for example grabbing onto objects quite natural. Wands usually offer either 3 or 6 degrees of freedom (DOF) tracking, meaning that either only the orientation or the orientation and position of the controller are tracked, respectively. 6 DOF controllers usually require a separate camera or other type of sensor to track the position of the wand, where as 3 DOF controllers can work independently.

The first mainstream wand controller was the Nintendo Wii Remote \cite{WiiRemoteMain}, offering 3 degrees of freedom. Wii Remotes use an accelerometer to track the orientation of the controller. The use of only an accelerometer introduces drift in the yaw direction, so the tracking system also includes an infrared transmitter bar. This transmitter bar is seen by a camera at the tip of the Wii Remote controller as two infrared dots. This information combined with the rest of the orientation data of the controller can be combined to eliminate yaw drift and get accurate rotational tracking. Conversely, when the infrared transmitter bar is not seen by the controller, i.e. the controller is pointed away from the screen, rotation tracking accuracy will deteriorate. The Wii Remote was augmented later on with the Wii MotionPlus add-on, which includes a gyroscope, improving the rotational tracking considerably. 

Sony's PlayStation Move \cite{PSMoveMain} quickly followed the Wii Remote. The Move controller features 6 DOF tracking, which makes it more usable in virtual reality applications requiring precise positional data. The positional tracking is achieved by having a separate camera track a glowing orb at the end of the controller. This kind of optical tracking results in the system being very vulnerable to occlusion issues. The user cannot for example hold the controller behind his body without losing accuracy. The range of the camera is also quite small, so the user does not have that big of an area to move in. \hl{more stuff}

The Razer Hydra is, like the PS Move, a 6 DOF controller. The Hydra, however, uses magnetic tracking, which removes the occlusion problems of the camera-based tracking in the Move. The system consists of two wired controllers that are connected to a magnetic central unit. Magnetic tracking has a shorter range, though, and since the controllers are wired this range is enforced very concretely. This makes the hydra best suited for use while sitting at a desk. When testing RUIS we mounted the magnetic unit onto the users torso in order to make the Hydra more mobile. This works quite well as long as the mount is stable. 

\section{Similar toolkits}
\label{section:similar}

\hl{RUIS for Processing}

\hl{MiddleVR}


%%%%%%%%%%%%%%%%%%%%%%%%%%%%%%%%%%%%%%%%%%%%%%%%%%%%%%%%%%%%%%%%%%
%%%%%%%%%%%%%%%%%%%%%%%%%%%%%%%%%%%%%%%%%%%%%%%%%%%%%%%%%%%%%%%%%%


\chapter{Design}
\label{chapter:design}

This chapter details the design considerations in the development of RUIS.

\section{Unity3D}
\label{section:unity3d}

Unity3D \cite{UnitySite} is a game development environment and an engine that has recently gained a lot of popularity because of its power and ease of use. Unity3D simplifies common game development and content creation tasks, shortening development time.

Code architecture in Unity3D is very much based on a common base class for all scene objects called GameObject. Functionality can be added to these GameObjects by attaching Components; either the ones that are built into Unity3D or by creating your own scripts inheriting the class MonoBehaviour. This way, different types of objects in the scene can be composed of small modular components with minimal coupling, keeping the overall structure quite easy to follow.

Unity3D also features prefabs: prefabricated objects that can be drag and dropped into a scene. This allows users to create pre-built common game objects and easily share them between scenes. The use of prefabs is an important part of using Unity3D efficiently, so it was a big point to take into consideration in the development of RUIS. 

\section{System Architechture}
\label{section:systemarchitecture}

\section{External Libraries}
\label{section:externallibraries}

Several external libraries and frameworks were used in the development of RUIS, especially to interface with all the different input devices that are supported. 

\subsection{OpenNI}
\label{subsection:external:openni}

RUIS uses OpenNI \hl{OpenNI site} as its skeletal tracking library. OpenNI is an open-source framework used for developing applications featuring natural bodily interaction. In the case of RUIS, OpenNI is used in tandem with the NiTE middleware libraries \hl{NiTE site}. NiTE extracts the skeletal data and RUIS references it through the OpenNI Unity Wrapper. 

\subsection{Oculus SDK}
\label{subsection:external:oculussdk}

\subsection{Other}
\label{subsection:external:other}

\hl{PSMoveWrapper & PSMoveClient} \\

\hl{Razer Hydra} \\

\hl{CSML}


%%%%%%%%%%%%%%%%%%%%%%%%%%%%%%%%%%%%%%%%%%%%%%%%%%%%%%%%%%%%%%%%%%
%%%%%%%%%%%%%%%%%%%%%%%%%%%%%%%%%%%%%%%%%%%%%%%%%%%%%%%%%%%%%%%%%%


\chapter{Display Configuration Management}
\label{chapter:displayconfigurationmanagement}

This chapter details all the implemented features regarding the management of displays. For starters, it is important for the user to be able to switch easily between monoscopic and stereoscopic modes of display. In addition to this, multiple displays of possibly different resolutions should be supported in a straightforward manner.

When displaying virtual worlds on a static screen relative to the movement of the observing user, the stereoscopic effect and general immersion level can be improved by modifying the perspective projection matrix to account for the changes in head position. This can make it seem as if the user were looking at the virtual world through a real window. This method requires accurate head tracking in order to be fully immersive.

As the last step in many projector systems, a projector keystoning correction will have to be applied. Projector keystoning is caused by projectors being off-center relative to the plane they are projecting on. This results in a distorted image that will be especially disturbing in multi-projector stereoscopic environments: The images for both eyes will not match and discrete seams will appear between the images meant for separate walls.

Finally, this chapter discusses the challenges we faced when integrating the Oculus SDK into our project. A head-mounted display requires some extra configurability for example in the head tracking settings. We also wanted to make sure this freedom did not come at the expense of usability.

\section{Multi-Display Systems and Stereoscopy}
\label{section:multidisplaysystems}

Many virtual reality applications make use of more than one display. Separate displays can be used to either show completely different information, or to wrap the virtual world around the user. Especially the latter is common in CAVE-like environments.

The way Unity3D handles multiple camera viewpoints is by requiring the application to specify the normalized viewport rectangle of each camera, in other words, the area of the application window that the camera will render to. While the user can do this manually for each camera, the process will become tedious very quickly, especially when configuring multiple stereoscopic displays. 

We tackle this problem by introducing a display manager component, RUISDisplayManager, to handle the bulk of display-related configuration issues. The user can add, remove and rearrange displays via the editor. Separate displays are called RUISDisplays, signifying their connection to RUIS. Each RUISDisplay works independently of the other displays - the display manager handles all shared tasks. Each RUISDisplay is then linked to a RUISCamera - RUIS's version of the Unity3D standard Camera. This chain of components implements all basic tasks required for multi-display, mono- and/or stereoscopic applications.

\subsection{Multiple Displays}
\label{subsection:multidisplaysystems:multipledisplays}

For each display, the user is able to specify the resolution of the display. The resolution specified is used to calculate the total needed resolution of the application. After this, each display, and the camera linked to it, is assigned its normalized viewport rectangle. 

\hl{could have example pic of different display configurations}

\subsection{Stereoscopy}
\label{subsection:multidisplaysystems:stereoscopy}

Once the basic framework for multiple displays is in place, it is quite straightforward to extend its functionality to include switching between mono and stereo displays. Most of the functionality for this is implemented at the RUISCamera level.

In addition to the Unity3D Camera attached to the RUISCamera, two new GameObjects with the Camera script will have to be added underneath it: the left and right eye cameras. The Camera script attached to the parent GameObject is treated as the monoscopic camera, while the left and right cameras are used for stereoscopic rendering. The RUISCamera script will then reference all these three Camera scripts and enable them based on whether the display rendered to is mono or stereo. 



\section{Perspective Projection for a Head-Tracked View}
\label{section:perspectiveprojection}



\section{Projector Keystoning Correction}
\label{section:keystonecorrection}

\cite{TUGrazKeystoning}

\section{Integrating Oculus Rift}
\label{section:integratingoculusrift}

Separate from all of the previous functinality is the Oculus Rift. Applications developed for the Rift require very different features from the display management system than CAVE applications, for example. Applications designed for head-mounted displays (HMD) usually use the HMD as their only display. 

%%%%%%%%%%%%%%%%%%%%%%%%%%%%%%%%%%%%%%%%%%%%%%%%%%%%%%%%%%%%%%%%%%
%%%%%%%%%%%%%%%%%%%%%%%%%%%%%%%%%%%%%%%%%%%%%%%%%%%%%%%%%%%%%%%%%%


\chapter{Input Management}
\label{chapter:inputmanagement}

\section{Coordinate Systems}
\label{section:coordinatesystem}

All of the motion control devices supported by RUIS operate in their own coordinate systems. In order to use all of these devices together the coordinate systems will have to be calibrated into one common system. Unity3D uses a consistent left-handed coordinate system with the basic distance unit being one meter, therefore, to make the use of RUIS as easy as possible for the end-user, it makes sense to transform all the input coordinates into the Unity3D coordinates.

\subsection{Coordinate System Calibration}
\label{subsection:coordinatesystems:calibration}

The coordinate system calibration process is implemented in its own standalone scene. The user can start this procedure via the RUIS Menu. 

In order to combine Move and Kinect coordinates samples are taken from both systems, after which a transformation matrix is calculated. This matrix transforms the Move coordinates into their Kinect counterparts. The matrix includes rotation and translation transforms. Scale is excluded since the scaling difference between the two coordinate systems is already known beforehand: Kinect uses millimeters, while Move uses centimeters.

The Move to Kinect transformation matrix is calculated by creating the equation \ref{eq:m2kmatrix}, where K and M are n by 3 matrices, containing n corresponding Kinect and Move samples, and A is the wanted transform. This linear matrix equation can then be solved straightforwardly.

\begin{equation}
    K = AM
    \label{eq:m2kmatrix}
\end{equation}

After finding the matrix A, the rotation and translation components are separated into a Quaternion and Vector3, which are ready to be applied to Move orientations and positions to turn them into their Kinect counterparts.

Also, the real-world floor location and normal are found during the calibration process by using OpenNI's scene analysis tools. This information is then used to tilt the coordinate system to counteract the tilt angle of the Kinect camera and to set the floor to be at the y = 0 plane if the user wishes to do so.

\subsection{Controlling Virtual Characters}
\label{subsection:coordinatesystems:virtualcharacters}

When moving virtual characters it is important to differentiate which coordinate system the character will be moved in. In addition to the local and global coordinates of the game object itself, we have to take tracker coordinates into account.

\section{Skeleton Tracking and Character Animation}
\label{section:skeletontracking}

To display a virtual character that matches the real-world movements of the user, the user skeleton will have to be tracked. RUIS uses OpenNI and a depth-sensing camera to achieve this. This procedure is detailed in chapter \ref{subsection:external:openni}. This section focuses on the software building upon the basic skeletal tracking to enable the creation of rich, interactive applications. 

\subsection{Using Premade Models and Armatures}
\label{subsection:skeletontracking:premademodels}

Hierarchical models

Scaling bones

Fine-tuning the model

Damping

\subsection{Combining Kinect and Mecanim}
\label{subsection:skeletontracking:kinectandmecanim}

\subsection{Adding Physics to a Kinect-Controlled Virtual Character}
\label{subsection:skeletontracking:physics}

Combining real-world physics with virtual physics requires some special consideration. Without special hardware, nothing that happens physically in the virtual world can affect how the user behaves in the real world, while real-world physics are always present in kinect-controlled motion: The floor and objects restrict movement, gravity affects the user's behavior, etc. Therefore, it is important to specify the wanted physical relationship for each application on a case-by-case basis. Also, in order to work around the limitations that the Kinect tracking imposes on the physical interaction, some precautions will have to be taken.



\section{Wand Controllers}
\label{section:wandcontrollers}

\subsection{PlayStation Move}
\label{subsection:wandcontrollers:psmove}

\subsection{Razer Hydra}
\label{subsection:wandcontrollers:hydra}

\subsection{Skeleton Wand}
\label{subsection:wadcontrollers:skeleton}



%%%%%%%%%%%%%%%%%%%%%%%%%%%%%%%%%%%%%%%%%%%%%%%%%%%%%%%%%%%%%%%%%%
%%%%%%%%%%%%%%%%%%%%%%%%%%%%%%%%%%%%%%%%%%%%%%%%%%%%%%%%%%%%%%%%%%


\chapter{Evaluation}
\label{chapter:evaluation}

\section{Final Toolkit}
\label{section:finaltoolkit}

\section{RUIS on the T-111.5400 Virtual Reality Course}
\label{section:vrcourse}

\section{TurboTuscany}
\label{section:turbotuscany}

For the launch of the RUIS we decided to create a demo using the Oculus Rift Tuscany demo as a base. This would guarantee recognizability while adding some interesting features that showcase the power of RUIS. 


%%%%%%%%%%%%%%%%%%%%%%%%%%%%%%%%%%%%%%%%%%%%%%%%%%%%%%%%%%%%%%%%%%
%%%%%%%%%%%%%%%%%%%%%%%%%%%%%%%%%%%%%%%%%%%%%%%%%%%%%%%%%%%%%%%%%%


\chapter{Conclusions}
\label{chapter:conclusions}


% Load the bibliographic references
% ------------------------------------------------------------------
% You can use several .bib files:
% \bibliography{thesis_sources,ietf_sources}
\bibliography{ref}


% Appendices go here
% ------------------------------------------------------------------
% If you do not have appendices, comment out the following lines
%\appendix
% \input{appendices.tex}

%\chapter{First appendix}
%\label{chapter:first-appendix}

%This is the first appendix. You could put some test images or verbose data in an
%appendix, if there is too much data to fit in the actual text nicely.

%For now, the Aalto logo variants are shown in Figure~\ref{fig:aaltologo}.

%\begin{figure}
%\begin{center}
%\subfigure[In English]{\includegraphics[width=.8\textwidth]{aalto-logo-en}}
%\subfigure[Suomeksi]{\includegraphics[width=.8\textwidth]{aalto-logo-fi}}
%\subfigure[Pä svenska]{\includegraphics[width=.8\textwidth]{aalto-logo-se}}
%\caption{Aalto logo variants}
%\label{fig:aaltologo}
%\end{center}
%\end{figure}


% End of document!
% ------------------------------------------------------------------
% The LastPage package automatically places a label on the last page.
% That works better than placing a label here manually, because the
% label might not go to the actual last page, if LaTeX needs to place
% floats (that is, figures, tables, and such) to the end of the
% document.
\end{document}
